\documentclass[]{article}
\usepackage{lmodern}
\usepackage{amssymb,amsmath}
\usepackage{ifxetex,ifluatex}
\usepackage{fixltx2e} % provides \textsubscript
\ifnum 0\ifxetex 1\fi\ifluatex 1\fi=0 % if pdftex
  \usepackage[T1]{fontenc}
  \usepackage[utf8]{inputenc}
\else % if luatex or xelatex
  \ifxetex
    \usepackage{mathspec}
  \else
    \usepackage{fontspec}
  \fi
  \defaultfontfeatures{Ligatures=TeX,Scale=MatchLowercase}
\fi
% use upquote if available, for straight quotes in verbatim environments
\IfFileExists{upquote.sty}{\usepackage{upquote}}{}
% use microtype if available
\IfFileExists{microtype.sty}{%
\usepackage{microtype}
\UseMicrotypeSet[protrusion]{basicmath} % disable protrusion for tt fonts
}{}
\usepackage[margin=1in]{geometry}
\usepackage{hyperref}
\hypersetup{unicode=true,
            pdftitle={Code zur Masterarbeit},
            pdfauthor={Xuan Son Le (4669361)},
            pdfborder={0 0 0},
            breaklinks=true}
\urlstyle{same}  % don't use monospace font for urls
\usepackage{color}
\usepackage{fancyvrb}
\newcommand{\VerbBar}{|}
\newcommand{\VERB}{\Verb[commandchars=\\\{\}]}
\DefineVerbatimEnvironment{Highlighting}{Verbatim}{commandchars=\\\{\}}
% Add ',fontsize=\small' for more characters per line
\usepackage{framed}
\definecolor{shadecolor}{RGB}{248,248,248}
\newenvironment{Shaded}{\begin{snugshade}}{\end{snugshade}}
\newcommand{\KeywordTok}[1]{\textcolor[rgb]{0.13,0.29,0.53}{\textbf{#1}}}
\newcommand{\DataTypeTok}[1]{\textcolor[rgb]{0.13,0.29,0.53}{#1}}
\newcommand{\DecValTok}[1]{\textcolor[rgb]{0.00,0.00,0.81}{#1}}
\newcommand{\BaseNTok}[1]{\textcolor[rgb]{0.00,0.00,0.81}{#1}}
\newcommand{\FloatTok}[1]{\textcolor[rgb]{0.00,0.00,0.81}{#1}}
\newcommand{\ConstantTok}[1]{\textcolor[rgb]{0.00,0.00,0.00}{#1}}
\newcommand{\CharTok}[1]{\textcolor[rgb]{0.31,0.60,0.02}{#1}}
\newcommand{\SpecialCharTok}[1]{\textcolor[rgb]{0.00,0.00,0.00}{#1}}
\newcommand{\StringTok}[1]{\textcolor[rgb]{0.31,0.60,0.02}{#1}}
\newcommand{\VerbatimStringTok}[1]{\textcolor[rgb]{0.31,0.60,0.02}{#1}}
\newcommand{\SpecialStringTok}[1]{\textcolor[rgb]{0.31,0.60,0.02}{#1}}
\newcommand{\ImportTok}[1]{#1}
\newcommand{\CommentTok}[1]{\textcolor[rgb]{0.56,0.35,0.01}{\textit{#1}}}
\newcommand{\DocumentationTok}[1]{\textcolor[rgb]{0.56,0.35,0.01}{\textbf{\textit{#1}}}}
\newcommand{\AnnotationTok}[1]{\textcolor[rgb]{0.56,0.35,0.01}{\textbf{\textit{#1}}}}
\newcommand{\CommentVarTok}[1]{\textcolor[rgb]{0.56,0.35,0.01}{\textbf{\textit{#1}}}}
\newcommand{\OtherTok}[1]{\textcolor[rgb]{0.56,0.35,0.01}{#1}}
\newcommand{\FunctionTok}[1]{\textcolor[rgb]{0.00,0.00,0.00}{#1}}
\newcommand{\VariableTok}[1]{\textcolor[rgb]{0.00,0.00,0.00}{#1}}
\newcommand{\ControlFlowTok}[1]{\textcolor[rgb]{0.13,0.29,0.53}{\textbf{#1}}}
\newcommand{\OperatorTok}[1]{\textcolor[rgb]{0.81,0.36,0.00}{\textbf{#1}}}
\newcommand{\BuiltInTok}[1]{#1}
\newcommand{\ExtensionTok}[1]{#1}
\newcommand{\PreprocessorTok}[1]{\textcolor[rgb]{0.56,0.35,0.01}{\textit{#1}}}
\newcommand{\AttributeTok}[1]{\textcolor[rgb]{0.77,0.63,0.00}{#1}}
\newcommand{\RegionMarkerTok}[1]{#1}
\newcommand{\InformationTok}[1]{\textcolor[rgb]{0.56,0.35,0.01}{\textbf{\textit{#1}}}}
\newcommand{\WarningTok}[1]{\textcolor[rgb]{0.56,0.35,0.01}{\textbf{\textit{#1}}}}
\newcommand{\AlertTok}[1]{\textcolor[rgb]{0.94,0.16,0.16}{#1}}
\newcommand{\ErrorTok}[1]{\textcolor[rgb]{0.64,0.00,0.00}{\textbf{#1}}}
\newcommand{\NormalTok}[1]{#1}
\usepackage{graphicx,grffile}
\makeatletter
\def\maxwidth{\ifdim\Gin@nat@width>\linewidth\linewidth\else\Gin@nat@width\fi}
\def\maxheight{\ifdim\Gin@nat@height>\textheight\textheight\else\Gin@nat@height\fi}
\makeatother
% Scale images if necessary, so that they will not overflow the page
% margins by default, and it is still possible to overwrite the defaults
% using explicit options in \includegraphics[width, height, ...]{}
\setkeys{Gin}{width=\maxwidth,height=\maxheight,keepaspectratio}
\IfFileExists{parskip.sty}{%
\usepackage{parskip}
}{% else
\setlength{\parindent}{0pt}
\setlength{\parskip}{6pt plus 2pt minus 1pt}
}
\setlength{\emergencystretch}{3em}  % prevent overfull lines
\providecommand{\tightlist}{%
  \setlength{\itemsep}{0pt}\setlength{\parskip}{0pt}}
\setcounter{secnumdepth}{0}
% Redefines (sub)paragraphs to behave more like sections
\ifx\paragraph\undefined\else
\let\oldparagraph\paragraph
\renewcommand{\paragraph}[1]{\oldparagraph{#1}\mbox{}}
\fi
\ifx\subparagraph\undefined\else
\let\oldsubparagraph\subparagraph
\renewcommand{\subparagraph}[1]{\oldsubparagraph{#1}\mbox{}}
\fi

%%% Use protect on footnotes to avoid problems with footnotes in titles
\let\rmarkdownfootnote\footnote%
\def\footnote{\protect\rmarkdownfootnote}

%%% Change title format to be more compact
\usepackage{titling}

% Create subtitle command for use in maketitle
\newcommand{\subtitle}[1]{
  \posttitle{
    \begin{center}\large#1\end{center}
    }
}

\setlength{\droptitle}{-2em}
  \title{Code zur Masterarbeit}
  \pretitle{\vspace{\droptitle}\centering\huge}
  \posttitle{\par}
  \author{Xuan Son Le (4669361)}
  \preauthor{\centering\large\emph}
  \postauthor{\par}
  \predate{\centering\large\emph}
  \postdate{\par}
  \date{8/13/2018}


\begin{document}
\maketitle

\begin{Shaded}
\begin{Highlighting}[]
\NormalTok{dfCensus <-}\StringTok{ }\KeywordTok{read.csv}\NormalTok{(}\StringTok{'c.csv'}\NormalTok{)[,}\OperatorTok{-}\DecValTok{1}\NormalTok{]}
\NormalTok{dfSurvey <-}\StringTok{ }\KeywordTok{read.csv}\NormalTok{(}\StringTok{'s.csv'}\NormalTok{)[,}\OperatorTok{-}\DecValTok{1}\NormalTok{]}

\NormalTok{dfCensus <-}\StringTok{ }\NormalTok{dfCensus[,}\KeywordTok{colnames}\NormalTok{(dfCensus) }\OperatorTok\StringTok{ }\KeywordTok{colnames}\NormalTok{(dfSurvey)]}

\NormalTok{dfSurvey <-}\StringTok{ }\KeywordTok{data.frame}\NormalTok{(}\StringTok{"cuadrantes"}\NormalTok{ =}\StringTok{ }\NormalTok{dfSurvey[,}\StringTok{"cuadrantes"}\NormalTok{],}
\NormalTok{                       dfSurvey[,}\KeywordTok{colnames}\NormalTok{(dfSurvey) }\OperatorTok\StringTok{ }\KeywordTok{colnames}\NormalTok{(dfCensus)])}
\end{Highlighting}
\end{Shaded}

\begin{Shaded}
\begin{Highlighting}[]
\CommentTok{# Anteil Missing Values}

\NormalTok{missingSurvey <-}\StringTok{ }\KeywordTok{data.frame}\NormalTok{(}\StringTok{"Variable"}\NormalTok{ =}\StringTok{ }\KeywordTok{colnames}\NormalTok{(dfSurvey), }\StringTok{"MissingSurvey"}\NormalTok{ =}\StringTok{ }\KeywordTok{t}\NormalTok{(}\KeywordTok{data.frame}\NormalTok{(}\KeywordTok{lapply}\NormalTok{(dfSurvey, }\ControlFlowTok{function}\NormalTok{(x) }\KeywordTok{round}\NormalTok{(}\DecValTok{100}\OperatorTok{*}\KeywordTok{sum}\NormalTok{(}\KeywordTok{is.na}\NormalTok{(x))}\OperatorTok{/}\KeywordTok{length}\NormalTok{(x),}\DecValTok{1}\NormalTok{)))), }\DataTypeTok{row.names =} \OtherTok{NULL}\NormalTok{)}

\NormalTok{missingCensus <-}\StringTok{ }\KeywordTok{data.frame}\NormalTok{(}\StringTok{"Variable"}\NormalTok{ =}\StringTok{ }\KeywordTok{colnames}\NormalTok{(dfCensus), }\StringTok{"MissingCensus"}\NormalTok{ =}\StringTok{ }\KeywordTok{t}\NormalTok{(}\KeywordTok{data.frame}\NormalTok{(}\KeywordTok{lapply}\NormalTok{(dfCensus, }\ControlFlowTok{function}\NormalTok{(x) }\KeywordTok{round}\NormalTok{(}\DecValTok{100}\OperatorTok{*}\KeywordTok{sum}\NormalTok{(}\KeywordTok{is.na}\NormalTok{(x))}\OperatorTok{/}\KeywordTok{length}\NormalTok{(x),}\DecValTok{1}\NormalTok{)))), }\DataTypeTok{row.names =} \OtherTok{NULL}\NormalTok{)}

\NormalTok{missingAll <-}\StringTok{ }\KeywordTok{left_join}\NormalTok{(missingSurvey, missingCensus, }\DataTypeTok{by =} \StringTok{'Variable'}\NormalTok{)}

\NormalTok{missingDF <-}\StringTok{ }\NormalTok{missingAll[}\KeywordTok{which}\NormalTok{(missingAll[,}\DecValTok{2}\NormalTok{] }\OperatorTok{>}\StringTok{ }\DecValTok{10} \OperatorTok{|}\StringTok{ }\NormalTok{missingAll[,}\DecValTok{3}\NormalTok{] }\OperatorTok{>}\StringTok{ }\DecValTok{10}\NormalTok{),]}
\KeywordTok{row.names}\NormalTok{(missingDF) <-}\StringTok{ }\KeywordTok{c}\NormalTok{(}\DecValTok{1}\OperatorTok{:}\KeywordTok{nrow}\NormalTok{(missingDF))}

\KeywordTok{xtable}\NormalTok{(missingDF)}
\end{Highlighting}
\end{Shaded}

\begin{verbatim}
## % latex table generated in R 3.5.0 by xtable 1.8-2 package
## % 
## \begin{tabular}{rlrr}
##   \hline
##  & Variable & MissingSurvey & MissingCensus \\ 
##   \hline
## 1 & edo\_civ & 23.80 & 25.40 \\ 
##   2 & subordinado & 62.60 & 66.10 \\ 
##   3 & independiente & 62.60 & 66.10 \\ 
##   4 & trab\_formal & 26.90 & 14.20 \\ 
##   5 & preslab & 63.10 & 75.70 \\ 
##   6 & jub & 0.00 & 25.60 \\ 
##    \hline
## \end{tabular}
\end{verbatim}


\end{document}
